\documentclass[hyperref={pdfpagelabels=false}]{beamer}
%\documentclass[handout]{beamer}
\let\Tiny=\tiny
\hypersetup{pdfpagemode=FullScreen}
\usepackage[ngerman]{babel}
\usepackage[utf8]{inputenc}
\usepackage{graphics}
\usepackage{listings}
\usepackage{verbatim}
\usepackage{uml}

%\setbeamertemplate{navigation symbols}{}
\usetheme{Boadilla}

\usecolortheme{beaver}
\usefonttheme{professionalfonts}
\useinnertheme{rounded}
\useoutertheme{smoothbars}
%\useoutertheme{sidebar}


\definecolor{lGray}{gray}{.90}

\newcommand{\code}[1]{\colorbox{lGray}{\texttt{#1}}}

\author{Christian Kniep}

\begin{document}

\title{python - OOP}  
\institute[ICAT Bandung]{Internation Center of Applied Technologies Bandung}
\date[\today]{\today} 

\begin{frame}
	\titlepage
\end{frame} 

\begin{frame}
	\frametitle{Table of content}
	\tableofcontents
\end{frame} 

        
\section{Own library}
    \subsection{test.py}
        \begin{frame}[fragile]{my 1st library}
			\begin{itemize}
                \item<1-> To use an object within the prompt we could import our script
                \item<1-> Create following little testobject in \code{test.py}:
                    \begin{lstlisting}
class o(object):
        def __init__(self): pass
        def __str__(self): return "huhu"
\end{lstlisting}
                \item<2-> make it executable \code{chmod +x test.py}
                \item<2-> import and use it:
                \begin{lstlisting}{language=python}
$ python
>>> import test
>>> o1=test.o()
>>> print o1
huhu
\end{lstlisting}
            \end{itemize}
		\end{frame}

\section{Controll-Structures}
    \subsection{if}
        \begin{frame}[fragile]{if-else-elif}
        	\begin{itemize}
                \item<1-> You know the construct if-else:
                    \begin{lstlisting}
if var:
    print 'not empty'
else:
    print 'empty'
\end{lstlisting}
                \item<2-> Instead of else you could check another condition
                    \begin{lstlisting}
if var1:
    print 'var1 not empty'
elif var2:
    print 'var2 not empty'
else:
    print 'both empty'
\end{lstlisting}             
            \end{itemize}
		\end{frame}
    \subsection{for}
        \begin{frame}[fragile]{for i in sth...}
        	\begin{itemize}
                \item<1-> To iterate over something you use \code{for}:
                    \begin{lstlisting}
if item in [1,2,3,4]:
    print item
\end{lstlisting}
            \end{itemize}
		\end{frame}

\section{Experience} 
	\subsection{Car}
		\begin{frame}{Car extensions}
			\begin{enumerate}
				\item<1-> Catch up the trail of the car-example from last session
                \item<2-> Add the function \code{driving(km)}, \code{refuel(liter)}
                \item<2-> The showCar are not able to drive, include this in your program
                \item<3-> Create an object for some load which has a function getWeight()
                \item<4-> Add a trunk which arranges the load-objects in a list
                \item<5-> Create \code{load(sth)} and \code{unload(sth)}
            \end{enumerate}
		\end{frame}


\end{document}
