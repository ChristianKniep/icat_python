%\documentclass[hyperref={pdfpagelabels=false}]{beamer}
\documentclass[handout]{beamer}
\let\Tiny=\tiny
\hypersetup{pdfpagemode=FullScreen}

\include{pythonlisting}

\usepackage[ngerman]{babel}
\usepackage[utf8]{inputenc}
\usepackage{graphics}
\usepackage{listings}
\usepackage{verbatim}
%\setbeamertemplate{navigation symbols}{}

\usetheme{Boadilla}

\usecolortheme{beaver}
\usefonttheme{professionalfonts}
\useinnertheme{rounded}
\useoutertheme{smoothbars}
%\useoutertheme{sidebar}

\definecolor{lGray}{gray}{.90}

\newcommand{\code}[1]{\colorbox{lGray}{\texttt{#1}}}
\author{Christian Kniep}

\begin{document}
\title[python - easy deploying]{python - easy deploying}  
\institute[ICAT Bandung]{Internation Center of Applied Technologies Bandung}
\date[\today]{\today} 

\begin{frame}
	\titlepage
\end{frame} 

\begin{frame}
	\frametitle{Table of content}
	\tableofcontents
\end{frame} 


\section{Introduction} 
	\subsection{Low vs High Level}
		\begin{frame}[fragile]
			\frametitle{Once upon a time...}
			\begin{itemize}
				\item<1-> 'ugly' assembler had to be used (e.g. MIPS)
                \begin{verbatim}
.data # define some varbiables
# the string we want printed
out:  .asciiz "Hello World"
 
.text # the program
main: li $v0, 4	  # cmd-reg to cmd 4 ('print')  
      la $a0, out # set out as the 1st arg 
      syscall     # execute the cmd
      li $v0, 10  # cmd-reg to cmd 10 ('exit')
      syscall     # execute the cmd
\end{verbatim}
            \end{itemize}
		\end{frame}
		\begin{frame}
			\frametitle{don't get me wrong}
			\begin{itemize}
				\item<1-> Assembler is the closest,directest way to program a CPU
                \item[$\Rightarrow$]<2-> so its the fastest code you can write
                \item<3-> you might just want to switch on the coffee-maker
                \item[$\Rightarrow$]<4-> you don't care if it takes 2 nano- or 200 milliseconds
            \end{itemize}
        \end{frame}
		\begin{frame}[fragile]
			\frametitle{High Level Programming}
			\begin{itemize}
            	\item<1-> it doesn't mean you have to have a high level to program it
                \item<2-> it the opposite: you only have to handle abstracted commands \\
                          the assembler code will be created for you
                \item<3-> checkout 'hello world' in C
                \begin{lstlisting}[language=c]
#include <stdio.h>

main()
{
 printf("Hello World \n");
}
\end{lstlisting}
            \end{itemize}
		\end{frame}
		\begin{frame}[fragile]
			\frametitle{python-Programming}
			\begin{itemize}
            	\item<1-> python is even more high level then c
                \begin{itemize}
                    \item<2-> doesn't care what type you are using
                    \item<3-> the syntax is intuitive
                    \item<4-> its an interpreter language, so you don't have to compile it
                    \item<4-> it's build from scratch, so there is no(t much) historical payload
                \end{itemize}
                \begin{lstlisting}[language=python]
print "hello world"
\end{lstlisting}
            \end{itemize}
		\end{frame}

    \subsection{Interpreter / Scripting Languages} 
		\begin{frame}[fragile]
			\frametitle{whats this suppose to mean?}
			\begin{itemize}
                \item<1-> you don't have to create an binary file
                \item<2-> the programm is evaluated linewise
                \item[$\Rightarrow$]<3-> so you are able to get a prompt
                \begin{lstlisting}{language=python}
$ python
>>> print "Hello World"
Hello World
\end{lstlisting}
                \item<3-> So if you see \textbf{\textgreater \textgreater \textgreater} I am in live-mode
            \end{itemize}
		\end{frame}
\section{Step By Step}
    \subsection{Executables}
		\begin{frame}[fragile]{Hello World \#2}
			\begin{itemize}
                \item<1-> We need a header
                \begin{lstlisting}{language=python}
#! python
# -*- coding: utf-8 -*-

print "Hello World"
\end{lstlisting}
                \item<2-> The header simply says what language do we use and
                \item<3-> that we use utf8-coding to allow some characters.
            \end{itemize}
		\end{frame}
		\begin{frame}[fragile]{Execute it}
			\begin{itemize}
                \item<1-> If you just have a textfile (without execute-permission)
                \begin{lstlisting}
$ python helloWorld.py
Hello World
\end{lstlisting}
                \item<2-> Due to the header the OS will know that it should use python:
                \begin{lstlisting}
$ chmod +x helloWorld.py
$ ./helloWorld.py
Hello World
\end{lstlisting}
            \end{itemize}
		\end{frame}
    \subsection{Strings}
		\begin{frame}[fragile]{Work with strings}
			\begin{itemize}
                \item<1-> concatenate
                \begin{lstlisting}
>>> print "1" + "1"
11
\end{lstlisting}
                \item<2-> Strings with linebreaks
\begin{lstlisting}
>>> x= """
1st line
2nd line
"""
>>> print x
1st line
2nd line
\end{lstlisting}
            \end{itemize}
		\end{frame}
		\begin{frame}[fragile]{Work with strings}
			\begin{itemize}
                \item<1-> multiply
                \begin{lstlisting}
>>> print "X"*10
XXXXXXXXXX
\end{lstlisting}
                \item<2-> formated output
\begin{lstlisting}
>>> print "decimal: %d" % 1
decimal: 1
>>> print "dec: %d - %d - %d" % (1,2,3)
1 - 2 - 3
>>> print "str: %s" % "some string"
str: some string
\end{lstlisting}
            \end{itemize}
		\end{frame}
		\begin{frame}[fragile]{Work with strings}
			\begin{itemize}
                \item<1-> more formated output
\begin{lstlisting}
>>> print "%-10s # %-10s" % ("Christian","Kniep")
Christian  # Kniep
>>> print "%-10s # %-10s" % ("Han","Solo")
Han        # Solo
\end{lstlisting}
            \item<2-> good for useability, clearly arranged
            \end{itemize}
		\end{frame}
		\begin{frame}[fragile]{some other commands}
			\begin{itemize}
                \item<1-> adjust strings
\begin{lstlisting}
>>> print hello.center(10)
  hello
>>> print "hello".center(10,"-")
--hello---
>>> print "hello".ljust(10,"-")
hello-----
>>> print "hello".rjust(10,"-")
-----hello
\end{lstlisting}
            \end{itemize}
		\end{frame}
		\begin{frame}[fragile]{some other commands}
			\begin{itemize}
                \item<1-> a quick glance of other stuff
\begin{lstlisting}
>>> len(X)
10
>>> x = "Hallo Welt"
>>> len(x)
10
>>> x.find("l")
2
>>> x.count("l")
3
...
\end{lstlisting}
            \end{itemize}
		\end{frame}
%%%%
\section{Comparisons}
    \subsection{Algebra}
		\begin{frame}{Basic Algebra}
			\begin{itemize}
                \item<1-> 3 operators included
                \begin{itemize}
                    \item[AND]<2-> True and True = True
                    \item<2-> True and False = False
                    \item[OR]<3-> True or True = True
                    \item<3-> True or False = True
                    \item<3-> False or False = False
                    \item[NOT]<3-> not True = False
                    \item<3-> not False = True
                \end{itemize}
                \item<4-> mix em! \\
                        True and (False or not False) = ?
            \end{itemize}
        \end{frame}    
    %%%% 
    \subsection{Digits}
		\begin{frame}[fragile]{Digits}
			\begin{itemize}
                \item<1-> basic digits
\begin{lstlisting}
>>> 1==1
True
>>> 1!=1
False
>>> 1>1
False
>>> 1>=1
True
>>> 1<1
False
>>> 1<=1
True
\end{lstlisting}
            \end{itemize}
		\end{frame}
		\begin{frame}[fragile]{Strings}
			\begin{itemize}
                \item<1-> compare strings
\begin{lstlisting}
>>> "X"=="X"
True
>>> "X"<="X"
True
>>> "Y"<="X"
False
>>> "Hello World".startswith("H")
True
\end{lstlisting}
            \end{itemize}
		\end{frame}
		\begin{frame}[fragile]{types}
			\begin{itemize}
                \item<1-> compare types
\begin{lstlisting}
>>> type("X")
<type 'str'>
>>> type(1)
<type 'int'>
>>> type(1.0)
<type 'float'>
>>> type("X") == type(1)
False
\end{lstlisting}
            \end{itemize}
		\end{frame}
\section{Math}
    \subsection{Basics}
		\begin{frame}[fragile]{Hello World in calculation}
			\begin{itemize}
                \item<1-> Simple example...
                \begin{lstlisting}
>>> x = 1
>>> y = 8
>>> print x + y
9
\end{lstlisting}
                \item<1-> advanced example...
                \begin{lstlisting}
>>> sqrt(9)
Traceback (most recent call last):
  File "<stdin>", line 1, in <module>
NameError: name 'sqrt' is not defined
\end{lstlisting}
            \end{itemize}
		\end{frame}
\section{Librarys}
    \subsection{math}
		\begin{frame}[fragile]{endless opportunities}
			\begin{itemize}
                \item<1-> normal import
                \begin{lstlisting}
>>> import math
>>> math.sqrt(9)
3.0
\end{lstlisting}
                \item<2-> import exclusiv commands
                \begin{lstlisting}
>>> from math import sqrt
>>> sqrt(9)
3.0
\end{lstlisting}
                \item<3-> import all commands
                \begin{lstlisting}
>>> from math import *
>>> ceil(3.6)
4.0
\end{lstlisting}
            \end{itemize}
		\end{frame}
        
        
\end{document}
