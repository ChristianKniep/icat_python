\documentclass[hyperref={pdfpagelabels=false}]{beamer}
%\documentclass[handout]{beamer}
\let\Tiny=\tiny
\hypersetup{pdfpagemode=FullScreen}
\usepackage[ngerman]{babel}
\usepackage[utf8]{inputenc}
\usepackage{graphics}
\usepackage{listings}
\usepackage{verbatim}
\usepackage{uml}

%\setbeamertemplate{navigation symbols}{}
\usetheme{Boadilla}

\usecolortheme{beaver}
\usefonttheme{professionalfonts}
\useinnertheme{rounded}
\useoutertheme{smoothbars}
%\useoutertheme{sidebar}


\definecolor{lGray}{gray}{.90}

\newcommand{\code}[1]{\colorbox{lGray}{\texttt{#1}}}

\author{Christian Kniep}

\begin{document}

\title{python - OOP}  
\institute[ICAT Bandung]{Internation Center of Applied Technologies Bandung}
\date[\today]{\today} 

\begin{frame}
	\titlepage
\end{frame} 

\begin{frame}
	\frametitle{Table of content}
	\tableofcontents
\end{frame} 


\section{More!} 
	\subsection{Review}
		\begin{frame}[fragile]{Lets look on the class again}
			\begin{itemize}
				\item<1-> inheritance
                \begin{lstlisting}{language=python}
class myC(object):
    def __init__(self,a):
        self.val = a
    def __del__(self):
        del self.val
    def __str__(self):
        return "My Value is: '%s'" % self.val
    def func(self,b):
        self.val *= b
\end{lstlisting}
            \end{itemize}
		\end{frame}
        
    \subsection{MyInt}
		\begin{frame}{Do it on your own}
            \begin{itemize}
                \item<1-> create an object which includes the four basic calculation
            \end{itemize}
            \begin{figure}
                \includegraphics<1>[height=0.5\columnwidth]{pics/MyInt.png}%
            \end{figure}
        \end{frame}
    \subsection{List}
		\begin{frame}{know some recursive stuff}
            \begin{itemize}
                \item<1-> Create the ListElement-object and the begining of the list and then create a list
            \end{itemize}
            \begin{figure}
                \includegraphics<1>[height=0.5\columnwidth]{pics/List.png}%
            \end{figure}
        \end{frame}



\end{document}
